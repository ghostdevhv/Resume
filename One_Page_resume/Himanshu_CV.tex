%%%%%%%%%%%%%%%%%%%%%%%%%%%%%%%%%%%%%%%%%
% Plasmati Graduate CV
% LaTeX Template
% Version 1.0 (24/3/13)
%
% This template has been downloaded from:
% http://www.LaTeXTemplates.com
%
% Original author:
% Alessandro Plasmati (alessandro.plasmati@gmail.com)
%
% License:
% CC BY-NC-SA 3.0 (http://creativecommons.org/licenses/by-nc-sa/3.0/)
%
% Important note:
% This template needs to be compiled with XeLaTeX.
% The main document font is called Fontin and can be downloaded for free
% from here: http://www.exljbris.com/fontin.html
%
%%%%%%%%%%%%%%%%%%%%%%%%%%%%%%%%%%%%%%%%%

%----------------------------------------------------------------------------------------
%	PACKAGES AND OTHER DOCUMENT CONFIGURATIONS
%----------------------------------------------------------------------------------------

\documentclass[a4paper,10pt]{extarticle} % Default font size and paper size

\usepackage{fontspec} % For loading fonts
\defaultfontfeatures{Mapping=tex-text}
\setmainfont[SmallCapsFont = Fontin SmallCaps]{Fontin} % Main document font
\fontspec{[FontAwesome.otf]}


\usepackage{xunicode,xltxtra,url,parskip} % Formatting packages

\usepackage[usenames,dvipsnames]{xcolor} % Required for specifying custom colors

% \usepackage[big]{layaureo} % Margin formatting of the A4 page, an alternative to layaureo can be 
%\usepackage{fullpage}
\usepackage{geometry}
\geometry{a4paper,margin=1.25cm}
%\geometry{a4paper,left=20mm, top=20mm}
 %To reduce the height of the top margin uncomment: \addtolength{\voffset}{-1.3cm}

\usepackage{hyperref} % Required for adding links	and customizing them
%\definecolor{linkcolour}{rgb}{0,0.2,0.6} % Link color
\definecolor{linkcolour}{rgb}{0.3,0.3,0.3} % Link color
\hypersetup{colorlinks,breaklinks,urlcolor=linkcolour,linkcolor=linkcolour} % Set link colors throughout the document

\usepackage{titlesec} % Used to customize the \section command
\titleformat{\section}{\Large\scshape\raggedright}{}{0em}{}[\titlerule] % Text formatting of sections
\titlespacing{\section}{0pt}{0pt}{0pt} % Spacing around sections

\usepackage{multicol}
\setlength{\columnsep}{1cm}

\usepackage{textcomp}

\usepackage{fontawesome}

\begin{document}

\pagestyle{empty} % Removes page numbering

\font\fb=''[cmr10]'' % Change the font of the \LaTeX command under the skills section

%----------------------------------------------------------------------------------------
%	NAME AND CONTACT INFORMATION
%----------------------------------------------------------------------------------------


\begin{enumerate}
\begin{minipage}{0.2\linewidth}
 \normalsize \faGithub\ {\href{https://github.com/himanshuv1996}{himanshuv1996}}\\
 \normalsize  \faLinkedinSquare\ {\href{https://www.linkedin.com/in/himanshu-v}{himanshu-v}}
 \end{minipage}
 \begin{minipage}{0.5\linewidth}
  \normalsize\par{\centering{\huge Himanshu \textsc{Verma}}\par} % Your name
 \par{\centering\normalsize {\textsc{A-210, LBS Hall of Residence, IIT Kharagpur}}\hfill\par}
  \par{\centering\normalsize {\textsc{West Bengal, India - 721302}}\hfill\par}
 \end{minipage}
 \begin{minipage}{0.3\linewidth}
\normalsize \faEnvelope\ {\href{mailto:vhimanshu.iitkgp@gmail.com}{vhimanshu.iitkgp@gmail.com}}\\
\normalsize \faPhone\ {  +91 86092 73495}
\end{minipage}
 \end{enumerate}

%----------------------------------------------------------------------------------------
%	RESEARCH INTERESTS
%----------------------------------------------------------------------------------------

%\section{Research Interests}

%- Software Design\hfill
%- Image Processing\hspace{3.7cm} \\

%----------------------------------------------------------------------------------------
%	EDUCATION
%----------------------------------------------------------------------------------------

\section{\large{Education}}

\begin{tabular}{r|p{16cm}}	
\normalsize{2013-2018} & \small{B.Tech and M.Tech (Dual Degree) in} \textsc{\small{Computer Science and Engineering}}\\
\textsc{\normalsize{(Expected)}}&\small{\textbf{Indian Institute of Technology}, Kharagpur}\\
&\textbf{Courses: }\small{{Deep Learning, Machine Learning, Artificial Intelligence, Parallel \& Distributed Algorithms, Operating Systems, Database Management Systems, Computer Networks, Information Retrieval, Speech \& Natural Language Processing, Compilers, Software Engineering, Algorithms-I \& II, Discrete Structures}}\\

\vspace{-7pt}
&\\
\normalsize{2013}& \small{Class XII,} \textsc{}\textsc{\small{Central Board of Secondary Education (CBSE)}} \\
&\small{\textbf{Central Academy School}, Gwalior}\\


% %------------------------------------------------

% 2011 & Class X, \textsc{}\textsc{State Board (Madhya Pradesh)} \\
% &\normalsize\textbf{Bal Vihar High School}, Gwalior\\

%\end{tabular}

\end{tabular}

%----------------------------------------------------------------------------------------
%	WORK EXPERIENCE
%----------------------------------------------------------------------------------------
\section{\large{Work Experience/Internship}}

\begin{tabular}{r|p{16cm}}

\textsc{\normalsize{Jun 2016}} & \textbf{\normalsize{Building Knowledge Base using Unstructured Data}}\\
\textsc{\normalsize{May 2016}} & \small{Software Development Intern at} \textsc{Flipkart}\small{, Bangalore}\\

& \footnotesize{- Preprocessed the whole dataset which includes cleaning of dataset \& and restructuring it into specific manner.}\\
& \footnotesize{- Proposed a domain specific chunking grammar, used along with various open information extractor tools like ReVerb, Ollie, Stanford CoreNLP to extract the phrases from each sentence. Used Word2Vec model to produce word embeddings.}\\
& \footnotesize{- Used skip-thought vectors for finding semantically and syntactically same sentences and then used Hierarchical Agglomerative Clustering (HAC) method to cluster them.}\\
\multicolumn{2}{c}{} \\
\end{tabular}

%----------------------------------------------------------------------------------------
%	Academic Projects
%----------------------------------------------------------------------------------------

\section{\large{Academic Projects}}

\begin{tabular}{r|p{16cm}}

\textsc{\normalsize{Nov 2016}} & \textbf{\normalsize{Song Lyrics Generation using Neural Networks}}\\
% \textsc{\normalsize{Feb 2016}} & \small{\textbf{Guide: }\textmd{\href{http://cse.iitkgp.ac.in/~pabitra/}{Professor Pabitra Mitra}}}\\
\textsc{\normalsize{Aug 2016}} & \footnotesize{- Built a database of song lyrics and used tensorflow to create a Long Short Term Memory (LSTM) neural network that works on a word-level language model which learns the artists’ styles of writing, including words, rhymes, chorus, etc. }\\
& \footnotesize{- tf-idf variance and assonance rhyme density measures were used to evaluate the model.}\\
\multicolumn{2}{c}{} \\

\textsc{\normalsize{Apr 2016}} & \textbf{\normalsize{Link Prediction using Apache Spark}} \\

% \textsc{\normalsize{Feb 2016}} & \small{\textbf{Guide: }\textmd{\href{http://cse.iitkgp.ac.in/~pabitra/}{Professor Pabitra Mitra}}}\\
\textsc{\normalsize{Mar 2016}}  & \footnotesize{- Given a social graph, predict which links are expected to appear in the future. Used Logistic Regression (mllib library - Apache Spark) and Jaccard's similarity to calculate the probability of two unconnected nodes getting connected in the future.}\\
\multicolumn{2}{c}{} \\


\textsc{\normalsize{Mar 2016}} & \textbf{\normalsize{Academia - Course Management System}}\\
% \textsc{\normalsize{Feb 2016}} & \small{\textbf{Guide: }\textmd{\href{http://cse.iitkgp.ac.in/~pabitra/}{Professor Pabitra Mitra}}}\\
\textsc{\normalsize{Feb 2016}} & \footnotesize{- Built a complete course management system that supported authentication \& authorization, User Access Control for 4 different types of users, calendar support and all major features one can expect from a CMS including faculty management, course progression, self-evaluated tests etc. }\\
& \footnotesize{- The complete workflow was built using the LAMP stack. Twitter Bootstrap was utilised for making the site fully responsive.}\\
\multicolumn{2}{c}{} \\

% \textsc{Nov 2015} & \textbf{Identifying Code Switching in Tweets} \\
% \textsc{Aug 2015} & \small{\textbf{Guide: }\textmd{\href{http://cse.iitkgp.ac.in/~pawang/}{Professor Pawan Goyal}}}\\
% & \footnotesize{- Used CRF (Conditional Random Field) for language tagging.)}\\
% & \footnotesize{- Used Maximum-Entropy Model and Decision Tree Model by giving them some features to identifying the point is Code Switched or not.}\\
% & \footnotesize{- Test our data on more than (0.1 million + lines) considering all the language change points as Switch point.} \\
% \multicolumn{2}{c}{} \\

\textsc{\normalsize{Nov 2015}} & \textbf{\normalsize{Compiler for Tiny C (A Subset of C Language)}}\\
% \textsc {\normalsize{Aug 2015}} & \small{\textbf{Guide: }\textmd{\href{http://www.iitkgp.ac.in/fac-profiles/showprofile.php?empcode=SSmUZ}{Professor Partha Pratim Das}}}\\
\textsc {\normalsize{Aug 2015}} & \footnotesize{- Developed a TINY C compiler using compilers principles, techniques and tools. The tools used for development were Flex and Bison. The compiler was written entirely in C++ language.}\\
% & \footnotesize{- }\\
\multicolumn{2}{c}{} \\

\textsc{\normalsize{Apr 2015}} & \textbf{\normalsize{Restaurant Automation System}}\\
% \textsc {\normalsize{Mar 2015}} & \small{\textbf{Guide: }\textmd{\href{http://www.iitkgp.ac.in/fac-profiles/showprofile.php?empcode=SSmUZ}{Professor Partha Pratim Das}}}\\
\textsc {\normalsize{Mar 2015}} & \footnotesize{- Developed a software built using JAVA Swing for a Restaurant Automation System which handles and automates the requests of the management and customers. Documented the software(SRS,SA/SD,UML,Test-Suite),which involved using UML case tools.}\\
& \footnotesize{- Separate records were maintained for handling the tasks and SQL was used for interfacing with the database. Key features included managing stocks, generate monthly sales receipt etc.}\\
% & \footnotesize{- }\\
\multicolumn{2}{c}{} \\
\end{tabular}

%----------------------------------------------------------------------------------------
%	TECHNICAL SKILLS
%----------------------------------------------------------------------------------------

\section{\large{Technical Skills}}

\begin{tabular}{r|p{16cm}}
\textsc{\normalsize{Programming}} 
& {\itshape{\small{Proficient:}}} \small{C, C++} \\
& {\itshape{\small{Familiar with:}}} \small{Python, Java, JavaScript, Scala} \\
\textsc{\normalsize{Libraries/Databases}} & \small{Tensorflow, MySQL}\\
\textsc{\normalsize{Frameworks}} & \small{Apache Spark, BootStrap, AngularJS}\\
\textsc{\normalsize{Markup/Templating}} & \small{HTML, CSS, \LaTeX}\\
\textsc{\normalsize{Software \& Tools}} & \small{StarUML, Netbeans ID, Eclipse}\\
\textsc{\normalsize{Systems/Platforms}} & \small{Git, Microsoft Windows, Linux(Ubuntu)}\\
\end{tabular}

%----------------------------------------------------------------------------------------
%	SCHOLASTIC ACHIEVEMENTS
%----------------------------------------------------------------------------------------

\section{\large{Scholastic Achievements}}


\small{- Stood amongst top 2.6\% participants (rank 150) in the \textbf{Google APAC 2017 University Test} - Round B. Handle - \textbf{\textit{lannister}}}.\\
\small{- Team \textbf{\textbf{Curious\_moles}} qualified for the onsite round and stood amongst top 20\% participants in the \textbf{ACM-ICPC'16} held at Coimbatore.} \\
\small{- Ranked in Top 5\% (amongst 150,000 candidates) in Joint Entrance Examination conducted by Indian Institute of Technology.} 
%\small{- Ranked in the State-wise Top 1\% (amongst 70,000 candidates) in State level Engineering Competitive Exam (MP PET).}


%----------------------------------------------------------------------------------------

%\newpage
%----------------------------------------------------------------------------------------

\end{document}